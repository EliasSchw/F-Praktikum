%Diese LaTeX-Vorlage für Praktikumsprotokolle in Form einer Veröffentlichung für das 
%F-Praktikum wurde erstellt von Andreas Nuber EP II, Uni Würzburg.
%Es wurde das Koma-Skript verwendet und sollte somit installiert sein
%desweiteren sollten alle packages installiert sein, die mit \usepackage{} 
%eingebunden werden.
%Zum Testen dieser Vorlage wurde MikTex verwendet sowie TeXnicCenter als Editor
%beim Compilieren waren 0 Fehler, 1 Warnung und 3 zu volle/leere Boxen. Das ist ok :)

%Für das Erstellen einfach den sinnfreien Text, der zum ausfüllen genommen wurde 
%ersetzen. Was sonst noch verändert werden sollte steht in den Kommentaren!


\documentclass[a4paper,10pt,twocolumn]{scrartcl} %Koma-Skript-Äquivalent zu "article"

%Von Elias entfernte Packages: inputenc, scrpage2
%\usepackage{scrpage2}          %ermöglicht änderung der Kopf-/Fußzeile


\usepackage{german}            %macht deutsche überschriften
%\usepackage[latin1]{inputenc}  %man kann Sonderzeiche wie ü,ö usw direkt eingeben
\usepackage{amsmath}           %macht
\usepackage{amsfonts}          %       Mathe
\usepackage{amssymb}           %              mächtiger
\usepackage{graphicx}          %erlaubt Graphiken einzubinden (.eps für dvi und ps sowie .jpg für pdf)
\usepackage[T1]{fontenc}       %Zeichenbelegung der verwendeten Schrift
\usepackage{ae}                %macht schöneres ß
\usepackage{typearea}	         %ermöglicht änderung des Seitenspiegels
\usepackage{lastpage}          %lässt auf die Seienanzahl zugreifen
\usepackage[margin=10pt,font=small,labelfont=bf]{caption} %macht die Bildbeschriftungen richtig

%\renewcommand{\figurename}{Abb.}

%\pagestyle{scrheadings}        %sagt Koma-Skript, dass selbstdefiniers Kopfzeilen verwendet werden
\typearea{16}                  %stellt Seitenspiegel ein
\columnsep25pt								 %definiert Breite zwischen den zwei Spalten von \twocolumns

\renewcommand{\pnumfont}{%     %ändert die Schriftart der Seitennummerierung
\normalfont\rmfamily\slshape}  %ändert die Schriftart der Seitennummerierung 



\begin{document}

% Titel und Abstract über beide Spalten
\twocolumn[{\csname @twocolumnfalse\endcsname
\titlehead{
    \begin{tabular*}{\textwidth}[]{@{\extracolsep{\fill}}lr}
    Betreuer: Der Betreuer & \today\\
    \end{tabular*}
    }
\title{Solid State Optics}
\author{Lukas Hein, Elias Schwarzkopf}
\date{}
\maketitle
\vspace{-8ex}
\begin{abstract}                                                %Beginn des Abstracts
    Fusce urna magna,neque eget lacus. Maecenas justo urna, lacinia vitae, vesti. Cras erat. Aliquam pede. vulputate e dolor ac adipiscing amet bibendum nullam, massa lacus molestie ut libero nec, diam et, sodales eget, feugiat ullamcorper id tempore. Ac dolor ac adipiscing amet bibendum. Maecenas felis nunc, aliquam ac, consequat vitae, feugiat at, blandit vitae, euismod vel, nunc. Aenean ut erat ut nibh commodo suscipit. . Ac dolor ac adipiscing amet bibendum. Maecenas felis nunc, aliquam ac, consequat vitae, feugiat at, blandit vitae, euismod vel, nunc. Aenean ut erat ut nibh commodo suscipit.  
    \\ \\ 
    \\ 
    Versuchsdurchführung: ??. ?.????\\       %Datum ändern!
    Protokollabgabe: ?? ??                 %Datum ändern!
    \\ 
    \\ 
\end{abstract} % Abstract bleibt über beide Spalten
\vspace{-6ex} % Reduziert den Abstand nach dem Abstract
}]

\input{introduction}
\nopagebreak
\section{theory}
\subsection{Plane parallel layer}
The investigated semiconductor samples can in first approximation be described as a plane parallel 
layer with a thickness $d$ and refractive index $N_2$ which acts like a fabry-perot interferometer.
\begin{figure}[h]
    \centering
    \includegraphics[width=0.5\textwidth]{Fabry-Perot.png}
    \caption{Simplified illustration of the measured samples as a plane parallel plate
    with thickness $d$ and refractive index $N_2$. The surrounding air is described with a 
    refractive indes $N_1\approx1$ \cite{Gerthsen}}
    \label{fig:layer}
\end{figure}

\begin{Set-Up}
\section{Set-Up}
Lorem ipsum dolor sit amet
\end{Set-Up}
\section{Bibliography}
\begin{thebibliography}{99}
    \bibitem{Gerthsen}
    Meschede D., \emph{Gerthsen Physik}, Springer Verlag, 25th Edition, 2018. Modified by Lukas Hein
    and Elias Schwarzkopf.
\end{thebibliography}

\end{document}