%Diese LaTeX-Vorlage für Praktikumsprotokolle in Form einer Veröffentlichung für das 
%F-Praktikum wurde erstellt von Andreas Nuber EP II, Uni Würzburg.
%Es wurde das Koma-Skript verwendet und sollte somit installiert sein
%desweiteren sollten alle packages installiert sein, die mit \usepackage{} 
%eingebunden werden.
%Zum Testen dieser Vorlage wurde MikTex verwendet sowie TeXnicCenter als Editor
%beim Compilieren waren 0 Fehler, 1 Warnung und 3 zu volle/leere Boxen. Das ist ok :)

%Für das Erstellen einfach den sinnfreien Text, der zum ausfüllen genommen wurde 
%ersetzen. Was sonst noch verändert werden sollte steht in den Kommentaren!


\documentclass[a4paper,10pt,twocolumn]{scrartcl} %Koma-Skript-Äquivalent zu "article"

%Von Elias entfernte Packages: inputenc, scrpage2
%\usepackage{scrpage2}          %ermöglicht änderung der Kopf-/Fußzeile


\usepackage{german}            %macht deutsche überschriften
%\usepackage[latin1]{inputenc}  %man kann Sonderzeiche wie ü,ö usw direkt eingeben
\usepackage{amsmath}           %macht
\usepackage{amsfonts}          %       Mathe
\usepackage{amssymb}           %              mächtiger
\usepackage{graphicx}          %erlaubt Graphiken einzubinden (.eps für dvi und ps sowie .jpg für pdf)
\usepackage[T1]{fontenc}       %Zeichenbelegung der verwendeten Schrift
\usepackage{ae}                %macht schöneres ß
\usepackage{typearea}	         %ermöglicht änderung des Seitenspiegels
\usepackage{lastpage}          %lässt auf die Seienanzahl zugreifen
\usepackage[margin=10pt,font=small,labelfont=bf]{caption} %macht die Bildbeschriftungen richtig

%\renewcommand{\figurename}{Abb.}

%\pagestyle{scrheadings}        %sagt Koma-Skript, dass selbstdefiniers Kopfzeilen verwendet werden
\typearea{16}                  %stellt Seitenspiegel ein
\columnsep25pt								 %definiert Breite zwischen den zwei Spalten von \twocolumns

\renewcommand{\pnumfont}{%     %ändert die Schriftart der Seitennummerierung
\normalfont\rmfamily\slshape}  %ändert die Schriftart der Seitennummerierung 



\begin{document}

% Titel und Abstract über beide Spalten
\twocolumn[{\csname @twocolumnfalse\endcsname
\titlehead{
    \begin{tabular*}{\textwidth}[]{@{\extracolsep{\fill}}lr}
    Betreuer: Der Betreuer & \today\\
    \end{tabular*}
    }
\title{Solid State Optics}
\author{Lukas Hein, Elias Schwarzkopf}
\date{}
\maketitle
\vspace{-8ex}
\begin{abstract}                                                %Beginn des Abstracts
    Fusce urna magna,neque eget lacus. Maecenas justo urna, lacinia vitae, vesti. Cras erat. Aliquam pede. vulputate e dolor ac adipiscing amet bibendum nullam, massa lacus molestie ut libero nec, diam et, sodales eget, feugiat ullamcorper id tempore. Ac dolor ac adipiscing amet bibendum. Maecenas felis nunc, aliquam ac, consequat vitae, feugiat at, blandit vitae, euismod vel, nunc. Aenean ut erat ut nibh commodo suscipit. . Ac dolor ac adipiscing amet bibendum. Maecenas felis nunc, aliquam ac, consequat vitae, feugiat at, blandit vitae, euismod vel, nunc. Aenean ut erat ut nibh commodo suscipit.  
    \\ \\ 
    \\ 
    Versuchsdurchführung: ??. ?.????\\       %Datum ändern!
    Protokollabgabe: ?? ??                 %Datum ändern!
    \\ 
    \\ 
\end{abstract} % Abstract bleibt über beide Spalten
\vspace{-6ex} % Reduziert den Abstand nach dem Abstract
}]

\section{Introduction}
Fourier Transform Infrared (FTIR) spectroscopy is a powerful method for determining
the electronic properties of metals and semiconductors. The FTIR technique is based on the
interaction of infrared light with the material. Reflection and transmission are used to determine 
important electronic properties. Especially for semiconductors, the FTIR method
is particularly useful because it helps to adjust parameters such as the doping concentration or
the bandgap. For building applications like LEDs, transistors, or solar cells, precise knowledge 
of the bandgap and doping concentrations is required. In contrast to other common methods, such as spectroscopy with
a monochromator, the FTIR method is much faster and more precise. Instead of varying the wavelength
through multiple measurements, the FTIR method can measure the entire spectrum at once due to
the use of a Michelson interferometer, which enables the interference of light. After Fourier transformation,
the complete spectrum is obtained in a single step.\\
Firstly, the theoretical basics of light-matter interaction and some important formulas regarding
reflection and transmission will be discussed. Following this theoretical introduction, the experimental
setup will be described. The final part of this paper will focus on data evaluation, starting with the
gas absorption of air. The second part of the data evaluation will involve determining the signal-to-noise
ratio. The last part will address the determination of several previously mentioned electronic and optical properties,
such as the refractive index, extinction coefficient, absorption coefficient, bandgap, and pulse matrix element
for all given semiconductor samples.
\section{theory}
Lorem ipsum dolor sit amet
\begin{Set-Up}
\section{Set-Up}
Lorem ipsum dolor sit amet
\end{Set-Up}
\section{Bibliography}
\begin{thebibliography}{99}
    \bibitem{Gerthsen}
    Meschede D., \emph{Gerthsen Physik}, Springer Verlag, 25th Edition, 2018. Modified by Lukas Hein
    and Elias Schwarzkopf.
\end{thebibliography}

\end{document}