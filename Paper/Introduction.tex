\section{Introduction}
Fourier Transform Infrared (FTIR) spectroscopy is a powerful method for determining
the electronic properties of metals and semiconductors. The FTIR technique is based on the
interaction of infrared light with the material. Reflection and transmission are used to determine 
important electronic properties. Especially for semiconductors, the FTIR method
is particularly useful because it helps to adjust parameters such as the doping concentration or
the bandgap. For building applications like LEDs, transistors, or solar cells, precise knowledge 
of the bandgap and doping concentrations is required. In contrast to other common methods, such as spectroscopy with
a monochromator, the FTIR method is much faster and more precise. Instead of varying the wavelength
through multiple measurements, the FTIR method can measure the entire spectrum at once due to
the use of a Michelson interferometer, which enables the interference of light. After Fourier transformation,
the complete spectrum is obtained in a single step.\\
Firstly, the theoretical basics of light-matter interaction and some important formulas regarding
reflection and transmission will be discussed. Following this theoretical introduction, the experimental
setup will be described. The final part of this paper will focus on data evaluation, starting with the
gas absorption of air. The second part of the data evaluation will involve determining the signal-to-noise
ratio. The last part will address the determination of several previously mentioned electronic and optical properties,
such as the refractive index, extinction coefficient, absorption coefficient, bandgap, and pulse matrix element
for all given semiconductor samples.