\section{Introduction}
Fourier Transformation Infra Red (FTIR) spectroscopy is a powerful method for determining
the electronic properties of metals and semiconductors. The FTIR technique is based on the
interaction of infrared light with the material. Reflection and transmission are used to determine 
important electronic properties. Especially for semiconductors the FTIR method
is particuarely useful because it helps to adjust parameters like the doping concentration or
the bandgap. For building applications like LED's, transistors or solar cells one need precise knowledge 
of bandgap and doping concentrations. In contrast to other common methods like spectroscopy with
a monochromator, the FTIR method is much faster and more precise. Instead of variing the wavelength
through various measurements, the FTIR method is able to measure the whole spectrum at once because
of a Michelson interferometer, which is used for the interference of the light. After fourier Transformation
one gets the complete spectrum at once.
Firstly, the theoretical basics of light matter interaction and some important formulas regarding
Reflection and Transmission will be discussed. After this theoretical introduction, the experimental
setup will be described. The last part of this paper will be the data evaluation, starting with the
gas absorption of air. The second part of the data evaluation will be the determination of the signal-to-noise
ratio. The last part will be the determination of several before mentioned electronic and optic properties
like the refractive index, extinction coefficient, absorption coefficient, band gap and pulse matrix element
for all given semiconductor samples.